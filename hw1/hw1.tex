% --------------------------------------------------------------
% Basic LaTeX template for homework assignments.
% COMS W4701 - Artificial Intelligence
% --------------------------------------------------------------
\documentclass[11pt]{article}

\usepackage[margin=1in]{geometry}
\usepackage{amsmath,amsthm,amssymb}
\usepackage{tabularx}
\usepackage[T1]{fontenc}
\usepackage{enumerate}
\usepackage[]{forest}

\forestset{.style={for tree=
{parent anchor=south, child anchor=north,align=center,inner sep=2pt}}}

\begin{document}

%           Your solutions start below this line
% --------------------------------------------------------------

\title{COMS W4119: Computer Networks\\
       Homework 1}
\author{Jackson Chen (jc4697)} % replace with your name and UNI
\maketitle

\section*{Bits on Wire}
  \begin{enumerate}[(a)]
    \item
      The largest IP datagram that can be transmitted over Ethernet is 1500 bytes. \footnote{https://en.wikipedia.org/wiki/Maximum\_transmission\_unit\#cite\_note-11}

      OC-1 has a transmission rate of 51.84 Mbit/s $= 6.48$ Mb/s \footnote{https://en.wikipedia.org/wiki/Optical\_Carrier\_transmission\_rates}

      Thus, the transmission delay is:
      \[ \frac{0.0015 \text{Mb}}{6.48 \text{Mb/s}} = 0.000231 \text{s} \]

      Since the speed of light in vacuum is $2.998 * 10^8$ m/s, the distance of the wire is:
      \[ 0.000231 \text{s} * 2.998 * 10^8 \text{m/s} * \frac{2}{3} = \boxed{46.2 \text{km}} \]
    \item
      The width of a packet on a wire becomes shorter because its transmission
      delay is shorter, allowing it to be sent out and received by the other
      end in a shorter period of time.
  \end{enumerate}

\section*{Circuit Switching and Packet Switching}
  \begin{enumerate}[(a)]
    \item
      Circuit switching provides bandwidth guarantees for resource intensive
      applications, such as audio/video apps, since it offers dedicated resources.
    \item
      Packet switching allows for better variation in use and better resource utilization, both in terms of the types of applications supported, frequency of use amongst users, and number of users.
    \item
      Two users
    \item
      For a specific user, that probability is $20\%$.

      However, if the question asks what the probability that some user is using
      the network (say within a $n$ person network), then that would be
      $1 - 0.8^n$
    \item
      The probability of $k$ users using the network simultaneously,
      and the others not using the network is:
      \[ 0.2^k * 0.8^{N-k} \]

      The number of $k$ user groups that can be selected from $N$ total users is:
      \[ {N \choose k}\]

      Thus, the overall probability is:
      \[ \boxed{0.2^k * 0.8^{N-k} * {N \choose k}} \]
    \item
      3 or more users must be using the network simultaneously in order to overwhelm
      the network. We will solve this problem by solving its inverse.

      Fraction of the time that 0 users are using the network simultaneously:
      \[ 0.8^{10} = 0.107 \]

      Fraction of the time that 1 user is using the network:
      \[ 0.2 * 0.8^9 * 10 = 0.268 \]

      Fraction of the time that 2 users are using the network simultaneously:
      \[ 0.2^2 * 0.8^8 * 45 = 0.302 \]

      Fraction of time that 3+ users are using the network (i.e network overwhelmed):
      \[ 1 - 0.107 - 0.268 - 0.302 = \boxed{0.323} \]
  \end{enumerate}

\section*{Message Switching \& Segmentation}
  \begin{enumerate}[(a)]
    \item
      I would propose a \textbf{packet-switching} design because it would support more
      users than a circuit switching alternative. Packet switching also supports a variety
      of applications, which is critical for this company, unlike circuit switching.
      Furthermore, audio and video streaming allows for some tolerance in packet loss,
      negating much of the downsides of packet switcing.
    \item
      Because there are $N$ routers between Alice and Bob, this means that there are $N+1$
      hops that the message must travel through.

      The time it takes for the message to travel to one node before continuing on is:
      \[ \frac{M}{R} \quad \text{seconds} \]

      Thus the total time for delivery is:
      \[ \frac{M(N+1)}{R} \quad \text{seconds} \]
    \item
      Across the trip, there is a transmission delay and a queueing delay that needs to be
      accounted for.

      Transmission delay: Since each packet is $\frac{M}{k}$ bits, the delay for each node is:
      \[ \frac{M}{k * R} \]

      Queueing delay: Initially, the first router is congested because all $k$ packets
      arrive. Due to transmission delay, queueing delay does not impact any of the other
      routers. The last packet transmitted has a queueing delay of:
      \[ \frac{M(k-1)}{k * R} \]

      Tracking the final packet, we can figure out what the time of delivery is.
      It undergoes the queueing delay of the first router, and then undergoes
      standard transmission delay for all of the routers in the path (incl. the link
      from the host machine):
      \[ \frac{M(k-1)}{k * R} + \frac{(N+1) * M}{k * R} = \boxed{\frac{M(N + k)}{k * R} \quad \text{seconds}} \]
    \item
      With the addition of the header, the total number of bits transferred will change.

      For the message switching scenario, the total time of delivery is:
      \[ \frac{(M + h)(N+1)}{R} \quad \text{seconds} \]

      For the packet switching scenario, the total time of delivery is:
      \[ \frac{(M + kh)(N + k - 1)}{k * R} \quad \text{seconds} \]

      The same end-to-end delay occurs between both mechanisms when
      \begin{align*}
        \frac{(M + h)(N+1)}{R} &= \frac{(M + kh)(N + k - 1)}{k * R} \\
        (M + h)(N + 1) &= \frac{M + kh}{k} (N + k - 1) \\
        (M + h)(N + 1) &= (\frac{M}{k} + h)(N + k - 1) \\
        MN + M + h(N + 1) &= \frac{MN}{k} + M - \frac{M}{k} + h(N + k - 1) \\
        MN - \frac{MN}{k} + \frac{M}{k} &= h(N + k - 1 - N - 1) \\
        \frac{MN(k-1)}{k} + \frac{M}{k} &= h(k-2) \\
        h &= \boxed{\frac{M(N(k-1)+1)}{k(k-2)}}
      \end{align*}

      With the above condition for h satisfied, both mechanisms have the same
      end-to-end delay.
    \item
      Message switching
    \item
      Both mechanisms get less efficient as the number of hops increase, but
      packet switching takes $\frac{M}{k * R}$ seconds longer per additional hop
      while message switching takes $\frac{M}{R}$ seconds longer, making
      packet switching the far more efficiently scalable mechanism.
    \item
      The transmission delay per node during packet switching is much lower,
      by a factor of $\frac{1}{k}$ (see part f). This is because in packet
      switching, the message is broken up into much smaller packets to be
      sent out, making the router wait less time to receive the complete datagram.
    \item
      The transmission delay for each 0.1MB packet is $\frac{1}{20}$ seconds.

      The amount of time it takes for a packet to transfer succesfully looks like
      the following:
      \begin{align*}
        &  \left( \frac{1}{10} \right)^0 * \frac{9}{10} * \frac{1}{20} && \text{Success on first try} \\
        +& \left( \frac{1}{10} \right)^1 * \frac{9}{10} * \frac{2}{20} && \text{Success on second try} \\
        +& \left( \frac{1}{10} \right)^2 * \frac{9}{10} * \frac{3}{20} && \text{Success on third try} \\
        +& ...
      \end{align*}

      This can be modeled as
      \[ \sum_{i=0}^{\infty} \left( \frac{1}{10} \right)^i * \frac{9}{10} * \frac{i+1}{20}
        = \frac{9}{10} \sum_{i=0}^{\infty} \left( \frac{1}{10} \right)^i * \frac{i+1}{20}
      \]

      Since there are a total of 10 packets that need to be transferred, the
      expected time for the file to be transferred is:
      \[ 10 * \frac{9}{10} \sum_{i=0}^{\infty} \left( \frac{1}{10} \right)^i * \frac{i+1}{20} = \boxed{9 \sum_{i=0}^{\infty} \left( \frac{1}{10} \right)^i * \frac{i+1}{20} \quad \text{seconds}} \]
    \item
      No
    \item
      The ideal delay is $1/2$ seconds. This number will be used as the baseline for
      the following parts.
      \begin{enumerate}[(i)]
        \item
          The threshold time would be: $0.5 * 1.2 = 0.6$ seconds.

          To meet that threshold, the file must succeed on its first try,
          hence $\boxed{0.9}$
        \item
          The threshold time would be: $0.5 * 2 = 1$ second.

          To exceed (or equal) the threshold, the file must fail at least once,
          hence $\boxed{0.1}$
        \item
          The threshold time would be: $0.5 * 1.2 = 0.6$ seconds.

          To meet that threshold, there must be at most two packet failures (note
          that the combination is added because all of the packets are
          indistinguishable, and thus failures could occur with any set of them)
          \begin{align*}
            & 0.9^{10} + 0.9^9 * 0.1 * {10 \choose 1} + 0.9^8 * (0.1)^2 * {10  \choose 2} \\
            =& 0.9^{10} + 0.9^9 * 0.1 * 10 + 0.9^8 * (0.1)^2 * 45 \\
            =& \boxed{0.93}
          \end{align*}
        \item
          The threshold time would be: $0.5 * 2 = 1$ second.

          To exceed (or equal) the threshold, there must be at least 10 packet failures.
          Since the first retransmission is always successful, then this means all
          of the packets must fail: $\boxed{0.1^{10}}$
      \end{enumerate}
    \item
      Segmentation significantly improves the performance of file transmission. Part (j)
      showed that segmentation not only increased the likelihood of file transfers
      meeting the $120\%$ threshold, but made it nearly impossible for it to
      exceed the $200\%$ threshold.

      The specific mechanism that achieves this is the lower transmission delay
      achieved when segmenting the file into packets. This makes any failure in
      the segmented case much less costly (for nonsegmented case, it could not
      tolerate any failures to reach the $120\%$ threshold).
    \item
      It would make the influence of segmentation much more significant in a positive
      way. By segmentating a large file into many small packets, each packet individually
      has a very low corruption rate, making it much more likely that the file would
      be transferred in its entirety without any changes. Meanwhile, sending the entire file at once would likely have a high corruption rate. Thus segmentation would have a
      higher influence with corruption rates taken into consideration.
  \end{enumerate}

\section*{Protocol Layers and Service Models}
  \begin{enumerate}[(a)]
    \item
      A layered internet architecture is important for two reasons:

      (1) Modularization makes maintenance, updating, and reuse of a system possible, especially given how
      distributed it is and vital its continuing uptime is.

      (2) It allows for identification and for relationship between the complex
      system's components.
    \item
      Layers may duplicate functionality with each other, increasing complexity
      and redundancy.
    \item
      The Internet protocol provides that all these services, if needed, must be
      implemented in the application layer.
    \item
      The host machine uses the application, transport, network, link, and physical layers.
    \item
      The router uses the network, link, and physical layers.
    \item
      The transport layer protocol (TCP vs UDP) does not affect the behavior of a switch,
      since a switch only operates at a link and physical level.
    \item
      A data segment is at the transport level, which means it contains information
      about the transport and application levels. With this in mind, here are the
      answers for the different parts:
      \begin{enumerate}[(i)]
        \item
          Determinable, since HTTP is in the application layer, which is encapsulated
          within a data segment.
        \item
          Determinable, since TCP is in transport layer, which is encapsulated
          within a data segment.
        \item
          Unclear, since IP is in the network layer, which is below the transport
          layer and thus out of scope for a data segment.
        \item
          Unclear, since Ethernet is in the physical layer, which is below the transport
          layer and thus out of scope for a data segment.
      \end{enumerate}
  \end{enumerate}

\end{document}
